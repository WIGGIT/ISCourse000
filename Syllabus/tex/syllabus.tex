% This syllabus template was created by:
% Brian R. Hall
% Associate Professor, Champlain College
% www.brianrhall.net

% Document settings
\documentclass[11pt]{article}
\usepackage[margin=1in]{geometry}
\usepackage[pdftex]{graphicx}
\usepackage{multirow}
\usepackage{setspace}
\pagestyle{plain}
\setlength\parindent{0pt}

\begin{document}

% Course information
\begin{tabular}{ l l }
  \multirow{3}{*}{\includegraphics[height=1.25in,width=1.25in]{logo_wwu.png}} & \LARGE CS160B \\\\
                                                                             & \LARGE Unix/Linux Shell Scripting \\\\
  & \LARGE Mon/Wed 1:10PM to 3PM \\\\
  & \LARGE Batmale 413\\\\
\end{tabular}
\vspace{10mm}

% Professor information
\begin{tabular}{ l l }
  % \multirow{6}{*}{\includegraphics[height=1.25in,width=1in]{logo_ccsf.png}} & \large Grace Woo \\\\
  & \large grwoo@ccsf.edu \\
  & \large Canvas Website: https://ccsf.instructure.com/courses/9154\\
  & \large Office Hours:  Mondays 3:30pm to 5pm in Batmale 456\\
\end{tabular}
\vspace{5mm}
\begin{center} Subject to changes throughout the semester.\\
\end{center}

% Course details
\textbf {\large \\ Course Description:} This is a course in basic Unix/Linux Shell Scripting. As such, the skills learned in this course will be applicable to any LInux or Unix system variant. The primary shell we use is the bash shell. Major topics covered include basic Unix tenets, commands, shell script arguments and variables, arguments, input and output as well as programming loops. This course requires a significant commitment of time. You should plan to spend an average of 2-5 hours of study and lab time for every hour spent in class for the duration of this course.\\

\textbf {Attendance Policy:} You are expected to attend all classes and be
seated for the class to begin promptly at ten minutes after the hour, when roll
will be taken. Any information that you miss due to nonattendance is solely
your responsibility. This may include helpful information for assignments and
tests. {\bf I may drop you from the class if you miss more than 2 classes in a
row without explanation.}\\

\textbf {Prerequisite(s):} None.

\textbf {Credit Hours:} 2 units\\
\textbf {Days:} Monday and Wednesdays: 1:10pm to 3:00pm in Batmale 413 (Ocean/Phelan campus)
Class will begin promptly at 1:10.\\
\textbf {\large Text(s):} \emph{The Linux Command Line}, Available Online for Free\\
\textbf {Author(s):} William Shotts;\\
\textbf {URL:} http://linuxcommand.org/tlcl.php \\

\textbf {\large Student Learning Outcome(s):} \\
At the completion of this course, students will be able to:
\begin{enumerate} \itemsep-0.4em
  \item Use command substitution to capture program output
  \item Use conditional statements to control the execution of shell scripts
  \item Write shell scripts to perform repetitive tasks using while and for loops
  \item Design and implement shell functions
  \item Identify and process command-line arguments
\end{enumerate}

% I recommend using \newpage here if necessary
\textbf {\large Grading Policy:} \\
\hspace*{40mm}
\begin{tabular}{ l l }
Assignments (3 Assignments)& 100pts/each \\
Labs (5 out of 7 Labs)& 20pts/each \\
Quizzes (3 Quizzes)& 100pts/each \\
\end{tabular} \\\\

You must notify me a few days before the test to request a make-up time for exams.  Electronic devices and computers may be used during all Quizzes.\\

\textbf {\large Final grades will be assigned on the following percentage scale:} \\\\
\hspace*{40mm}
\begin{tabular}{ l l | l l }
90\% - 100\% A \\
80\% - 89\% B \\
70\% - 79\% C \\
60\% - 69\% D \\
0\% - 59\% F \\
\end{tabular} \\

Students who do not take the final exam will be assigned a grade of "FW". An "FW" is an "F" grade that also indicates that the student did not complete the course.\\

% Course Policies. These are just examples, modify to your liking.

\textbf {\large Homework:}

The best way to learn how to program is to do it! Homework will be assigned
about once a week.

\begin{itemize}

	\item \textbf {Labs:}

        There are 7 Lab Exercises and many of them will be done in class. Each
        Lab Exercise is worth 20 points. Your top 5 Lab scores will count
        toward the final grade. Extra lab scores will be considered for extra
        credit. Late labs will not be accepted for full credit.  A late hand in
        may be considered for 50\% credit.

	\item \textbf {Assignments:}

        There are 3 graded Assignments during this course. Assignments must be
        handed in by the Assignment due date on Canvas. Assignments are worth
        100 points each. Late labs will not be accepted for full credit.
        A late hand in may be considered for 50\% credit.

    \end{itemize}

\textbf {Homework Lateness policy:}

\hspace{3mm}

        Because of the importance of keeping up with the pace of class, late
        homework will be penalized severely. All homework assignments are due
        by midnight the night of the due date specified. Late homework will be
        penalized 5\% if it is turned in before I go through the solution in
        class (the following class after it's due). Starting the day I present
        the solution, late homework will be penalized 50\%. You will get no
        credit for turning in my solution as your own. All homework you turn in
        must be your own, even after we have gone through a solution in class.

\hspace{3mm}

\textbf{Cheating}

\hspace{3mm}

Cheating of any kind will not be tolerated. It will result in a grade of 0 on
the assignment or test in question and can be cause for a failed grade and
disciplinary action, including suspension or expulsion. Getting help from
others is not cheating as long as you're not copying their work or allowing
them to copy yours. On the exams, any collaboration or copying constitutes
cheating.

\hspace{3mm}

\textbf{Software and Computer Access}

\hspace{3mm}

You can access hills either from a computer in the ACRC in Batmale Hall or
remotely using ssh. If you access hills from the ACRC you should use the linux
machines near the rear exit (see the next section). You may also login from a
Windows system, but you must first login to the ACRC Windows network. If you
wish to do this, you should take an orientation during the first week of class.
You can access hills remotely using ssh. Do not use telnet. The particulars of
remote access are your responsibility.  The server is hills.ccsf.edu. It is
your responsibility to get these issues worked out in order to complete your
assignments on time.  The ACRC holds a series of three orientation classes on
hills and on their Windows network. A schedule is posted in the ACRC. If you
are new to hills or the ACRC you should consider attending these sessions.

\hspace{3mm}

\textbf{Access to Linux Machines (Springfield Cluster)}

\hspace{3mm}

By enrolling in this course you will have an account on the linux machines.
These accounts will be created a week or so into the semester as announced in
class. The account name and initial password follow the same pattern as your
hills account. The linux machines are divided between those in the ACRC and
those in the linux classroom. They all share a common set of logins and a
common exported file system for home directories. Once on linux, you can use
ssh at the command-line to log in to hills to access the class public data
files.  For security reasons the linux machines are only accessible through
hills or another local machine. They are not registered via DNS. If you want to
reach a linux system from off-campus, you must login to hills and ssh using the
IP address of a linux machine. These IP addresses are taped to the linux
machines in the linux area of the ACRC. You should visit it and make a note of
them.

\hspace{3mm}

\textbf{Drop Procedures}

\hspace{3mm}

Generally it is your responsibility to drop or withdraw from a class by the
final deadlines given in your course schedule. Do not ask me to drop you; use
the Web4 system, or contact the Office of Admissions and Records to be
withdrawn from a class. If you have more than three unexplained consecutive
class absences, I may drop you from the class. If your name is on the roll at
the end of the semester and you have stopped attending class, you will be
assigned a final grade of FW. I will not give a late or retroactive drop or
withdrawal.

\hspace{3mm}

\textbf{Disability Accomodations}

\hspace{3mm}

Students with disabilities who need accommodations are encouraged to contact
the instructor. Disabled Students Programs and Services (DSPS) is available to
facilitate the reasonable accommodation process. The DSPS office is located in
the Rosenberg Library, Room 323 and can be reached at (415) 452-5481.

% Course Outline

\hspace{3mm}

\textbf {\large Tentative Course Outline}:

The weekly topic coverage might change as it depends on the progress of the class.

\begin{table}[h!]
  \normalsize % The size of the table text can be changed depending on content. Remove if desired.

\begin{tabular}{ | c | c | }
\hline
\textbf{Week Of} & \textbf{Homework Schedule} \\
\hline
10/23 & \begin{minipage}{.85\textwidth}
\begin{itemize} \itemsep-0.4em
	\vspace{1mm}
	\item Lab 1: Introductory Review
        \item Quiz 1 (Take Home): Covers Review Topics
	\vspace{1mm}
\end{itemize}
\end{minipage} \\
\hline
10/30 & \begin{minipage}{.85\textwidth}
\begin{itemize} \itemsep-0.4em
	\vspace{1mm}
	\item Lab 2: Command Substitution
        \item Assignment 1: Files and Users
        \item 11/1: Quiz 1 Due
	\vspace{1mm}
\end{itemize}
\end{minipage} \\
\hline
11/6 & \begin{minipage}{.85\textwidth}
\begin{itemize} \itemsep-0.4em
	\vspace{1mm}
        \item Lab 3: Arguments
	\vspace{1mm}
\end{itemize}
\end{minipage} \\
\hline
11/13 & \begin{minipage}{.85\textwidth}
\begin{itemize} \itemsep-0.4em
	\vspace{1mm}
        \item Lab 4: File Checker
        \item Assignment 2: Virtual Card
        \item 11/15: Assignment 1 Due
	\vspace{1mm}
\end{itemize}
\end{minipage} \\
\hline
11/20 & \begin{minipage}{.85\textwidth}
\begin{itemize} \itemsep-0.4em
	\vspace{1mm}
	\item Lab 5: Function Use
        \item 11/22 Quiz in Class Covers Through Week 4
	\vspace{1mm}
\end{itemize}
\end{minipage} \\
\hline
11/27 & \begin{minipage}{.85\textwidth}
\begin{itemize} \itemsep-0.4em
	\vspace{1mm}
	\item Lab 6: Nested Loops
        \item 11/29: Assignment 2 Due
	\vspace{1mm}
\end{itemize}
\end{minipage} \\
\hline
12/4 & \begin{minipage}{.85\textwidth}
\begin{itemize} \itemsep-0.4em
	\vspace{1mm}
	\item Lab 7: Read While
        \item Assignment 3: Virtual Card++
	\vspace{1mm}
\end{itemize}
\end{minipage} \\
\hline
12/11 & \begin{minipage}{.85\textwidth}
\begin{itemize} \itemsep-0.4em
	\vspace{1mm}
        \item In-Class Review
	\vspace{1mm}
\end{itemize}
\end{minipage} \\
\hline
12/18 & \begin{minipage}{.85\textwidth}
\begin{itemize} \itemsep-0.4em
	\vspace{1mm}
	\item 12/20: Final Quiz
        \item 12/20: Assignment 3 Due
	\vspace{1mm}
\end{itemize}
\end{minipage} \\
\hline
\end{tabular} 
\end{table}

\end{document}



